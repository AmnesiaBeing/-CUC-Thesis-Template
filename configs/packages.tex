% 首先引用这个文件,其余配置信息在具体文件中具体配置
% 该文件中包含大部分需要包含的功能包

\usepackage{geometry} % 布局
\usepackage{hyperref}      % 超链接
\usepackage{listings}      % 代码块
\usepackage{courier}       % 字体
\usepackage{fontspec}      % 字体
\usepackage{fancyhdr}      % 页眉页脚相关宏包
% 如果使用subfigure包,需要切换注释行
\usepackage{tocloft} % 目录
% \usepackage[subfigure]{tocloft} % 目录
\usepackage{amsmath,amsthm,amsfonts,amssymb,bm} %数学
\usepackage{graphicx}      % 图片
\usepackage{longtable,booktabs,tabularx} % 表格
\usepackage{ulem} % 下划线
\usepackage{multido} % 多次执行命令
\usepackage{titlesec} % 多级标题样式设置
\usepackage{bookmark} % 书签(目录页可跳转链接等)
\usepackage{pdfpages} % 插入pdf
\usepackage{subfig} % 使用子图
\usepackage{threeparttable} % 使用表格注释
\usepackage{diagbox} % 使用表格表头斜线分区

% 参考文献宏包,可能需要注意设置问题
\usepackage[
    style=gb7714-2015,
    % gbnamefmt=lowercase, % 引用外文论文作者名字是否需要小写
]{biblatex}
